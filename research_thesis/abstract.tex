\begin{abstract}

Agricultural market volatility poses significant challenges for farmers, traders, and policymakers in Sri Lanka, where vegetable prices fluctuate unpredictably due to complex interactions between weather patterns, supply dynamics, and economic factors. This research develops a machine learning-based forecasting system for predicting wholesale carrot prices at Dambulla market, integrating data from the Central Bank of Sri Lanka, Copernicus Climate Data Store, Ceylon Petroleum Corporation fuel prices, and market supply-demand indicators.

Comprehensive feature engineering across price patterns, weather conditions, supply factors, demand indicators, fuel costs, and temporal features was performed. A systematic 4-stage feature selection pipeline combining Random Forest importance, Mutual Information, correlation analysis, and multicollinearity removal reduced 163 engineered features to 9 optimal features while preserving predictive power and eliminating redundancy. Seven forecasting models including ARIMA, LSTM variants (Univariate, Multivariate, Simple, and Bidirectional), and Random Forest were rigorously evaluated using consistent train-validation-test splits and multiple performance metrics including MAPE, MAE, RMSE, and R\textsuperscript{2}.

Traditional ARIMA methods demonstrated fundamental limitations with MAPE exceeding 50\% for both univariate and multivariate specifications, validating the inadequacy of linear assumptions for complex agricultural markets. LSTM-based deep learning approaches achieved substantial improvements, with the optimized Simple LSTM model reaching 19.93\% MAPE and R\textsuperscript{2} of 0.8651 using only 9 carefully selected features, effectively capturing non-linear temporal dependencies while avoiding overfitting through architectural simplicity and aggressive feature selection. Systematic ablation studies quantified feature category contributions, revealing price history as the dominant predictor while weather, supply, and fuel factors provided meaningful incremental accuracy. SHAP analysis enhanced model interpretability, confirming theoretically expected negative rainfall-price relationships.

The research delivers a deployment-ready system integrating the best-performing model with a Retrieval-Augmented Generation AI agent using Groq API and natural language interface via Gradio, democratizing sophisticated forecasting for non-technical stakeholders. This work contributes replicable methodology for agricultural price forecasting, empirically validates deep learning superiority over traditional approaches for complex market dynamics, and provides operational tools supporting informed decision-making across the agricultural value chain in developing economies.

\textbf{Keywords:} carrot price prediction, LSTM neural networks, Dambulla market, agricultural forecasting, machine learning, weather-price relationships, RAG system, AI agent, time series analysis, Sri Lankan agriculture

\end{abstract}
