\chapter{Introduction}

\section{Research Overview}

Agriculture remains a vital component of Sri Lanka's economy. Among the various agricultural products, vegetables constitute an essential part of the domestic food supply chain. However, vegetable markets in Sri Lanka have historically been characterized by high price volatility, which creates substantial challenges for both farmers and consumers. Carrot, being one of the major vegetables traded in the country, experiences frequent and often unpredictable price fluctuations that can severely impact the economic stability of farming communities.

Daily price movements in the Dambulla market reflect complex interactions between multiple factors including weather patterns, fuel costs, regional supply levels, and market demand dynamics. Understanding and predicting these price movements has become increasingly important for stakeholders throughout the agricultural value chain.

This research investigates the application of machine learning techniques to forecast wholesale carrot prices in the Dambulla market. The study explores how climatic conditions, particularly precipitation patterns from major growing regions, along with economic factors such as fuel prices and supply indicators, influence daily price variations. Through the development and comparison of multiple forecasting models, this work aims to identify the most effective approach for short-term price prediction in the context of Sri Lankan vegetable markets.

\section{Problem Statement}

Vegetable price volatility presents serious problems for the agricultural sector in Sri Lanka. Farmers often struggle to make informed decisions about planting schedules, harvest timing, and market entry due to the uncertainty surrounding future prices. When prices drop unexpectedly, farmers may face losses that threaten their financial stability. Conversely, sudden price spikes can lead to affordability issues for consumers while also triggering market interventions that complicate the natural price discovery process.

Traditional approaches to understanding vegetable prices have relied primarily on historical averages and seasonal patterns. However, these methods fail to capture the complex, non-linear relationships between prices and the various factors that influence them. The lack of reliable forecasting tools means that decision-making across the supply chain remains largely reactive rather than proactive. This situation calls for more sophisticated analytical approaches that can better model the temporal dependencies and multi-dimensional influences inherent in agricultural price data.

The challenge is further complicated by the fact that different factors may have varying degrees of influence on prices at different times. For instance, heavy rainfall in major growing regions like Nuwara Eliya might have delayed effects on market prices as supply disruptions gradually propagate through the distribution network. Similarly, changes in fuel prices might impact transportation costs, which in turn affect wholesale pricing. Capturing these temporal lags and interactions requires modeling techniques that can learn from sequential data and account for multiple input variables simultaneously.

\section{Proposed Solution}

This research proposes a comprehensive machine learning framework for carrot price forecasting that addresses the limitations of traditional approaches. The framework employs three modeling approaches: ARIMA models for baseline time series analysis, LSTM neural networks to capture complex temporal patterns with external factors including precipitation, fuel prices, and supply indicators, and Random Forest regression as an ensemble learning alternative.

The framework incorporates systematic feature selection to identify the most relevant predictors from multiple data sources. Additionally, it includes an intelligent AI agent that provides natural language access to forecasting insights through a web interface, making sophisticated predictions accessible to non-technical stakeholders.

% Motivation section removed to streamline Introduction (moved details to Methodology/Literature Review)

\section{Background}

Sri Lanka's agricultural sector is central to the nation's economy and food security, with vegetable farming concentrated in the central highlands. The Nuwara Eliya district supplies a substantial portion of temperate vegetables including carrots. The Dambulla Economic Centre functions as the primary wholesale distribution point for vegetables, where prices established daily influence retail pricing throughout the country.

Carrot cultivation occurs year-round across main growing areas including Nuwara Eliya, Welimada, and Bandarawela, though production volumes vary seasonally. These regions experience different rainfall patterns and microclimates, theoretically providing continuous supply. However, the system is vulnerable to disruptions from excessive rainfall damaging crops, transportation challenges from fuel price increases, and market dynamics amplifying price volatility. Despite government efforts to collect market data, forecasting tools remain limited. This research addresses this gap by introducing sophisticated analytical capabilities.

\section{Research Objectives}

This research aims to develop an effective machine learning-based forecasting system for carrot price prediction in the Dambulla market. The study pursues three primary objectives:

\textbf{1. Develop Machine Learning Models for Price Forecasting:} Implement multiple forecasting models including ARIMA for baseline time series analysis, LSTM neural networks to capture temporal dependencies with external factors, and Random Forest as an ensemble alternative. Apply systematic feature selection to identify optimal predictors from historical price, meteorological, economic, and market data.

\textbf{2. Evaluate and Compare Model Performance:} Conduct comprehensive evaluation of all developed models using appropriate performance metrics to identify the most effective forecasting approach. Include systematic validation studies to assess model reliability and understand key factors influencing prediction accuracy.

\textbf{3. Build an AI Agent Using RAG System:} Develop a deployment-ready intelligent agent integrating the best-performing forecasting model with natural language processing capabilities to provide farmers, traders, and policymakers intuitive access to forecasting insights through a conversational interface.

\section{Scope of the Research}

This research focuses specifically on wholesale carrot prices in the Dambulla market, allowing in-depth analysis while maintaining manageable complexity. While the methodologies could potentially apply to other vegetables or markets, the empirical work concentrates on this single commodity and location.

The temporal scope encompasses January 2020 through July 2025, providing over five years of daily price observations. This period includes both normal market conditions and volatile periods (including Sri Lanka's recent economic challenges), ensuring models are exposed to diverse market scenarios.

The study investigates ARIMA, LSTM, and Random Forest models as representative samples spanning traditional statistical methods, deep learning, and ensemble learning. The RAG-based AI agent serves as a delivery mechanism for predictions, with the primary focus remaining on forecasting methodology itself.