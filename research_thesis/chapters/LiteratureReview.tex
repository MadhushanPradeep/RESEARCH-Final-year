\chapter{Literature Reviews}

This chapter reviews the existing literature on agricultural price forecasting, with particular emphasis on vegetable price prediction using various computational techniques. The review is organized into several thematic sections that progress from traditional statistical approaches to modern machine learning and deep learning methodologies. Each section examines relevant studies, highlighting their methodologies, findings, and contributions to the field. The chapter concludes by identifying gaps in current research and establishing the context for this study.

\section{Traditional Statistical Methods for Agricultural Price Prediction}

Traditional statistical approaches have long formed the foundation of agricultural price forecasting. Among these methods, AutoRegressive Integrated Moving Average (ARIMA) models have been extensively employed due to their ability to capture temporal dependencies in price series data.

Ruhunuge et al. (2024) conducted an econometric investigation into climate-driven carrot price variations in Sri Lanka using Vector Autoregression (VAR) modeling. Their study spanned twenty-three years (2000-2023) of wholesale carrot price data from the Hector Kobbekaduwa Agrarian Research and Training Institute (HARTI), combined with climate data from the Meteorology Department. The researchers applied first-order differencing to capture volatility patterns, resulting in 852 monthly observations that passed rigorous unit root tests. Their VAR model, optimized with a lag structure of six periods based on Akaike Information Criterion (AIC), revealed that precipitation changes significantly influenced carrot prices (p-value = 0.0447), while temperature demonstrated limited predictive value. The impulse response analysis showed that a one-unit standard deviation increase in precipitation resulted in an immediate 2.8\% increase in carrot prices, peaking at 1.2\% in the third interval before stabilizing. The derived model equation demonstrated that increased rainfall significantly lowered current carrot prices with a coefficient of -16.09719, while the positive lagged price coefficient of 0.141777 indicated price momentum effects \citep{ruhunuge2024}.

Chen et al. (2021) compared ARIMA against modern machine learning approaches for Malaysian agricultural commodity price prediction. Their ARIMA implementation utilized parameters (p=1.5, d=1, q=1) determined through Augmented Dickey Fuller Test and comprehensive Auto/Partial Correlation Function analysis. While ARIMA achieved remarkable average Mean Squared Error (MSE) of 0.251 for smaller datasets, with chili prediction reaching exceptional 0.027 MSE, the model experienced concerning 74.1\% performance degradation when confronted with increased data complexity. This finding highlighted a fundamental limitation of traditional statistical methods when dealing with large, complex agricultural datasets \citep{chen2021}.

These studies demonstrate that while traditional statistical methods provide interpretable models and perform well with smaller datasets under stable market conditions, they struggle with non-linear relationships and large-scale data, motivating the exploration of machine learning alternatives.

\section{Machine Learning Approaches for Crop Price Forecasting}

Machine learning techniques have gained prominence in agricultural price prediction due to their ability to capture complex, non-linear relationships between multiple variables without requiring explicit model specification.

\subsection{Tree-Based Ensemble Methods}

Ranaweera et al. (2023) investigated vegetable price predictability in Sri Lanka using data mining techniques. Their comprehensive study analyzed four vegetables (beans, eggplant, carrots, and pumpkins) from five economic centers using four-year historical data (2018-2021) from multiple institutions including the Central Bank, Department of Agriculture, Meteorological Department, and Ceylon Petroleum Corporation. The researchers evaluated five machine learning algorithms including Linear Regression, SMO Regression, Multilayer Perceptron, Random Forest, and M5P using 10-fold cross-validation in WEKA 3.8.6. Random Forest emerged as the superior model, achieving Mean Absolute Error (MAE) values ranging from 10.58 for pumpkins to 27.62 for beans. Pumpkins demonstrated the highest prediction accuracy exceeding 85\%, while beans presented the most challenging forecasting scenario. The study revealed that rainfall variability (0.00 to 53.40mm) and temperature fluctuations, alongside fuel prices affecting transportation costs, critically impacted price variations \citep{ranaweera2023}.

Choong et al. (2024) developed a Genetic Algorithm-Based Neural Network (GANN) approach within an Agricultural Knowledge Management System (AKMS) for Malaysia. Using monthly vegetable prices from 2010 to 2021 (144 observations), their GANN model achieved 98.40\% accuracy with MAPE of 1.6042\%, significantly outperforming both ARIMA (98.32\% accuracy) and SARIMA (98.37\% accuracy). The genetic algorithm optimization with 20 chromosomes, three hidden layers containing five nodes each with ReLU activation, and roulette wheel selection methodology enabled dynamic adaptation to agricultural market dynamics. The model particularly excelled in RMSE (0.06674) and MAE (0.5571) metrics, demonstrating superior handling of nonlinear relationships in seasonal price variations \citep{choong2024}.

\subsection{Support Vector Machines and Hybrid Approaches}

Kakulapati et al. (2022) explored vegetable price prediction against temperature changes using web scraping to collect real-time weather and price data from Hyderabad. Their comparison of Decision Tree Regression, Random Forest Regression, and Linear Regression revealed that Decision Tree Regression achieved superior accuracy in predicting prices based on temperature variations. The innovative web scraping methodology enabled dynamic data collection at five-day intervals, addressing limitations of static historical datasets. This real-time approach provided farmers with timely insights for cultivation decision-making based on weather-price correlations \citep{kakulapati2022}.

Bayona-Oré et al. (2021) conducted a comprehensive systematic review of machine learning applications in agricultural price prediction from 2011-2020. Their analysis revealed that Neural Network models were most frequently employed (24 algorithms), followed by statistical models (20 algorithms) and Support Vector Machines (9 occurrences). The review identified that all studies employed positivism paradigm with quantitative approaches, predominantly using supervised learning due to availability of labeled historical price data. Performance metrics analysis showed RMSE, MAPE, and MAE as the most commonly used evaluation measures. Geographically, China dominated with 11 studies examining 17 products, while India contributed 6 studies covering 12 products, indicating regional concentration in research efforts \citep{bayona2021}.

\section{Deep Learning Methods for Agricultural Time Series Forecasting}

Deep learning approaches, particularly recurrent neural networks and their variants, have revolutionized time series forecasting by effectively capturing long-term dependencies and complex temporal patterns.

\subsection{Long Short-Term Memory (LSTM) Networks}

Zhang et al. (2024) investigated short-term vegetable price forecasting for Beijing's wholesale markets using LSTM models. Their study utilized 14.7 years of daily price data (January 2009 to September 2023) from seven major wholesale markets, analyzing six representative vegetables from four categories. The LSTM architecture comprised two layers with 32 neurons each, optimized learning rate of 0.0027, dropout rate of 0.2, batch size of 500, and 200 training epochs using Adam optimizer. The model achieved exceptional performance with R² scores of 0.958 and MAE of 0.143, representing over 5\% improvement compared to CNN, XGBoost, and SVR. Vegetable-specific accuracy varied notably: celery (93.3\%), carrots (92.9\%), oyster mushrooms (90.2\%), and spiny cucumbers (90.1\%), with trend prediction concordance rates exceeding 70\% for most vegetables. Wilcoxon signed-rank tests confirmed statistically significant improvements over competing methods (p < 0.05) \citep{zhang2024}.

Yin et al. (2020) developed an innovative STL-ATTLSTM model integrating Seasonal Trend decomposition using Loess (STL) with attention mechanism-based LSTM for South Korean vegetable markets. Their research targeted five supply-and-demand-sensitive vegetables (cabbage, radish, onion, hot pepper, garlic) using data from January 2012 to December 2019. The sophisticated architecture employed STL to separate time series into trend, seasonality, and remainder components, with attention mechanism assigning dynamic weights to input variables during training. The model comprised an attention layer with softmax activation, LSTM layer with 6 cell units using tanh activation, dropout layer (0.2 rate), and fully connected layers, trained for 1000 epochs using Adam optimizer. The STL-ATTLSTM achieved exceptional average RMSE of 380 and MAPE of 7\%, representing 12\% higher prediction accuracy compared to attention LSTM without STL preprocessing. The model successfully eliminated the one-month prediction lag phenomenon common in highly volatile time-series data by utilizing STL remainder components rather than raw price data \citep{yin2020}.

\subsection{Hybrid Deep Learning Architectures}

Guo et al. (2022) proposed an innovative AttLSTM-ARIMA-BP hybrid model for corn price prediction in Sichuan Province, China. Using 511 weekly observations from March 2011 to April 2021, they employed Apriori association rule mining to identify 12 critical spatial-temporal factors across multiple provinces and related commodity prices. Their hybrid architecture strategically integrated Attention Mechanism for dynamic weight calculation, LSTM for non-linear temporal dependencies, ARIMA for linear trend modeling, and Back Propagation Neural Network for final prediction synthesis. The model achieved outstanding performance with MAPE of 0.0043, MAE of 1.51, RMSE of 1.642, and remarkable R² of 0.9992, significantly outperforming seven competing models including Linear Regression, Random Forest, XGBoost, LightGBM, single LSTM, multivariate LSTM, and AttLSTM. While traditional regression models maintained reasonable accuracy during stable periods, they failed dramatically during volatile market conditions, whereas the hybrid model consistently delivered accurate predictions regardless of price behavior patterns \citep{guo2022}.

Avinash et al. (2024) introduced Hidden Markov-based Deep Learning approaches for forecasting TOP (Tomato, Onion, Potato) commodity prices in India. Their research utilized 911 weekly price observations from Azadpur Mandi (January 2006 to June 2023), applying Hidden Markov Models (HMMs) for feature extraction to identify hidden states in price data. Optimal hidden states were determined through grid search: six states for tomato and eight states each for onion and potato. These hidden states served as inputs to four deep learning models: Multilayer Perceptron (MLP), Recurrent Neural Networks (RNN), Gated Recurrent Units (GRUs), and Long Short-Term Memory (LSTM). Extensive hyperparameter optimization across 126 combinations per model included batch sizes, epochs (200 with early stopping), hidden layers, and units. The hybrid HM-DL models achieved superior performance with RMSE reductions of 9.77-17.50\% for tomato, 15.02-44.39\% for onion, and 7.94-32.60\% for potato compared to baseline approaches. HM-RNN consistently emerged as the best performer for training data, while HM-LSTM excelled for tomato testing data due to superior long-memory capabilities in capturing significant price spikes. Diebold-Mariano tests confirmed statistically significant differences between hybrid and baseline models \citep{avinash2024}.

\section{Feature Engineering and Selection in Agricultural Forecasting}

Effective feature engineering and selection constitute critical components of successful agricultural price prediction models, as they determine which variables contribute most significantly to forecasting accuracy.

The reviewed studies employed diverse approaches to feature selection. Ranaweera et al. (2023) systematically incorporated four key factors: rainfall, temperature, fuel price, and crop production, demonstrating that climatic factors particularly influenced price variations in tropical agricultural systems. Their analysis revealed substantial variability in price predictability across vegetables, with pumpkins showing highest accuracy and beans presenting the most challenging scenario \citep{ranaweera2023}.

Guo et al. (2022) utilized Apriori association rule mining algorithm to identify 12 critical spatial-temporal factors influencing corn prices, including prices from multiple provinces and related commodities. This data-driven approach to feature discovery enabled their hybrid model to capture complex inter-commodity and inter-regional price relationships \citep{guo2022}.

Yin et al. (2020) demonstrated sophisticated feature engineering by incorporating meteorological variables (average temperature, minimum temperature, humidity, precipitation, temperature threshold days, typhoon advisories) specifically for main production areas during harvest periods. Their approach strategically focused harvest-time meteorological data for immediately marketed crops (cabbage, radish) while excluding weather factors for warehouse-stored crops (hot pepper, onion, garlic) with delayed market entry. Additionally, they integrated trading volume data as production proxies and import/export information for comprehensive market analysis \citep{yin2020}.

Chen et al. (2021) implemented a dual-experimental design: first utilizing univariate time-series data spanning 11 years, then incorporating multivariable features including temperature, humidity, precipitation, and crude oil prices. Their comparative analysis revealed that ARIMA excelled with smaller datasets while LSTM demonstrated 45.5\% improvement in MSE for larger, more complex datasets, highlighting the importance of matching model complexity to data characteristics \citep{chen2021}.

\section{Model Evaluation and Performance Metrics}

Rigorous model evaluation using appropriate performance metrics is essential for assessing prediction accuracy and comparing different forecasting approaches.

\subsection{Common Evaluation Metrics}

The literature review reveals widespread adoption of several key performance metrics. Mean Absolute Percentage Error (MAPE) emerged as the most prevalent metric, utilized by Zhang et al. (2024) achieving 0.143, Yin et al. (2020) achieving 7\%, Guo et al. (2022) achieving 0.0043, and Choong et al. (2024) achieving 1.6042\%. MAPE's popularity stems from its scale-independent nature and intuitive percentage interpretation \citep{zhang2024, yin2020, guo2022, choong2024}.

Root Mean Square Error (RMSE) and Mean Absolute Error (MAE) were also frequently employed for absolute error measurement. Ranaweera et al. (2023) comprehensively evaluated models using MAE, RMSE, Relative Absolute Error (RAE), and Root-Relative Square Error (RRSE), providing multi-dimensional performance assessment. Additionally, coefficient of determination (R²) was utilized to measure explained variance, with Zhang et al. (2024) achieving 0.958 and Guo et al. (2022) achieving exceptional 0.9992 \citep{ranaweera2023, zhang2024, guo2022}.

\subsection{Statistical Validation Techniques}

Several studies incorporated rigorous statistical validation beyond basic performance metrics. Avinash et al. (2024) employed Diebold-Mariano (DM) tests to establish statistically significant differences between hybrid and baseline models, ensuring observed improvements were not due to chance. Zhang et al. (2024) utilized Wilcoxon signed-rank tests to confirm LSTM's superior performance over competing methods with p-values below 0.05 threshold \citep{avinash2024, zhang2024}.

Ruhunuge et al. (2024) implemented comprehensive VAR model validation including stability condition testing (all characteristic roots within unit circle, largest root at 0.92), residual diagnostics showing no significant autocorrelation (p-values above 0.10), and heteroscedasticity testing (p-value = 0.15) indicating constant variance, thereby validating model reliability \citep{ruhunuge2024}.

Yin et al. (2020) conducted systematic time-step optimization through grid search over multiple lag values, determining optimal time-step of 4 for superior performance across most vegetables. This methodological rigor in hyperparameter selection contributed to their model's exceptional accuracy \citep{yin2020}.

\section{Regional Perspectives and Data Sources}

Agricultural price prediction research exhibits significant geographical concentration, with diverse data sources and regional considerations influencing methodology and applicability.

\subsection{Asian Agricultural Markets}

The majority of reviewed studies focused on Asian agricultural markets. China dominated with multiple studies: Guo et al. (2022) analyzed Sichuan corn prices using data from China's agricultural big data website and commodity exchanges, while Zhang et al. (2024) examined Beijing's seven major wholesale vegetable markets \citep{guo2022, zhang2024}. Bayona-Oré et al. (2021) confirmed this geographical concentration, identifying China as most researched country with 11 studies examining 17 products, followed by India with 6 studies covering 12 products \citep{bayona2021}.

South Asian markets received notable attention through studies in Sri Lanka and India. Ranaweera et al. (2023) and Ruhunuge et al. (2024) both focused on Sri Lankan vegetable markets, utilizing data from the Central Bank of Sri Lanka, Department of Agriculture, and Hector Kobbekaduwa Agrarian Research and Training Institute. Avinash et al. (2024) examined India's Azadpur Mandi in Delhi, one of Asia's largest wholesale markets, using data from Agmarknet spanning over 17 years \citep{ranaweera2023, ruhunuge2024, avinash2024}.

Southeast Asian perspectives emerged through Malaysian market studies. Chen et al. (2021) and Choong et al. (2024) both utilized data from Malaysia's Federal Agricultural Marketing Authority (FAMA), examining chicken, chili, tomato, and potato prices over extended periods. Their work addressed regional challenges including aging farming population and insufficient knowledge management systems \citep{chen2021, choong2024}.

\subsection{Data Collection Approaches}

Data sourcing strategies varied significantly across studies. Traditional approaches relied on official government databases and agricultural institutions. Multiple studies utilized meteorological data from national weather services: Ranaweera et al. (2023) from Sri Lanka's Meteorological Department, Yin et al. (2020) from Korean Meteorological Administration, and Chen et al. (2021) integrated climate data with price information \citep{ranaweera2023, yin2020, chen2021}.

Innovative data collection methods emerged in recent research. Kakulapati et al. (2022) employed web scraping to extract real-time weather and price data, collecting information every five days to create dynamic datasets reflecting current market conditions. This approach addressed limitations of static historical datasets, enabling timely insights for agricultural decision-making \citep{kakulapati2022}. Guo et al. (2022) utilized commodity exchange data alongside traditional agricultural databases, capturing futures market information for wheat and soybeans to inform spot price predictions \citep{guo2022}.

The temporal scope of datasets ranged considerably: from Ruhunuge et al.'s 23-year span (2000-2023) capturing long-term climate-price relationships, to Zhang et al.'s 14.7 years (2009-2023) of daily observations providing high granularity, to shorter focused studies like Ranaweera et al.'s four-year analysis (2018-2021) enabling rapid model development \citep{ruhunuge2024, zhang2024, ranaweera2023}.

\section{Application Domains and Practical Implementation}

Beyond academic contributions, several studies addressed practical implementation challenges and developed systems for real-world agricultural stakeholders.

Chen et al. (2021) developed a comprehensive web-based platform following Model-View-Controller (MVC) pattern using Django framework. Their system featured secure user authentication, interactive visualization dashboards with customizable forecast durations, commodity selection interfaces, downloadable CSV exports, and responsive design for compatibility across devices. The platform empowered farmers, government agencies, and agricultural stakeholders to make informed decisions regarding plantation planning, supply chain optimization, and policy formulation \citep{chen2021}.

Choong et al. (2024) integrated price forecasting within an Agricultural Knowledge Management System (AKMS) following DIKW (Data, Information, Knowledge, Wisdom) pyramid framework enhanced with IoT and Big Data capabilities. Their platform combined knowledge management principles with e-commerce functionality, supporting both explicit knowledge (documented procedures) and tacit knowledge (farmer experiences) through integrated information sharing. The system addressed Malaysia's National Agrofood Policy (NAP 2.0) 2021-2030 objectives for creating sustainable, technology-based agrofood industry \citep{choong2024}.

Practical applications focused on multiple stakeholder benefits. Zhang et al. (2024) provided week-ahead forecasts with detailed trend analysis revealing distinct fluctuation patterns for different vegetables, offering insights for growers, consumers, and policymakers. Avinash et al. (2024) emphasized helping farmers optimize storage decisions, identify favorable selling periods, and minimize losses through reliable price forecasting. Ranaweera et al. (2023) highlighted the importance of AI-driven forecasting for mitigating financial risks associated with price fluctuations in developing tropical economies \citep{zhang2024, avinash2024, ranaweera2023}.

\section{Research Gaps and Limitations}

While the reviewed literature demonstrates significant progress in agricultural price prediction, several gaps and limitations warrant attention for future research directions.

\subsection{Methodological Gaps}

Bayona-Oré et al. (2021) identified lack of epistemological consideration in most studies, with all employing positivism paradigm and quantitative approaches without exploring alternative philosophical frameworks. Their review revealed absence of comprehensive model comparison frameworks and limited exploration of model interpretability, despite increasing emphasis on explainable AI in agricultural applications \citep{bayona2021}.

Feature selection approaches remained largely empirical rather than systematic. While Guo et al. (2022) employed data mining for feature discovery and Yin et al. (2020) strategically selected features based on agricultural domain knowledge, most studies lacked rigorous statistical feature selection procedures such as mutual information analysis, recursive feature elimination, or ablation studies to quantify individual feature contributions \citep{guo2022, yin2020}.

The challenge of model generalization across different regions, vegetables, and market conditions received limited attention. Chen et al. (2021) noted concerning 74.1\% performance degradation of ARIMA with increased complexity, while studies generally focused on specific vegetables or regions without investigating cross-commodity or cross-market applicability \citep{chen2021}.

\subsection{Data-Related Limitations}

Data availability constraints significantly influenced research scope. Bayona-Oré et al. (2021) observed that agricultural product selection was primarily driven by data availability rather than economic importance or market significance. Geographic concentration in China and India reflected both research capacity and data infrastructure availability, while other developing agricultural economies remained understudied \citep{bayona2021}.

Temporal granularity varied across studies, with some utilizing daily data, others weekly or monthly observations. The impact of temporal resolution on prediction accuracy and practical applicability remained underexplored. Additionally, most studies relied on historical price data without incorporating real-time market signals or social media sentiment that might capture emerging market trends.

External factor integration remained incomplete in many studies. While weather and fuel prices received attention, broader macroeconomic indicators (exchange rates, inflation, policy changes), consumer behavior patterns, and supply chain disruptions (as experienced during COVID-19) were generally absent from modeling frameworks.

\subsection{Practical Implementation Challenges}

The gap between research models and operational deployment systems remained substantial. While Chen et al. (2021) and Choong et al. (2024) developed web-based platforms, most studies concluded with model performance evaluation without addressing deployment challenges such as model updating procedures, computational requirements, user interface design, or stakeholder adoption barriers \citep{chen2021, choong2024}.

Model interpretability and explainability received insufficient attention despite their importance for farmer acceptance and trust. Agricultural stakeholders need to understand not only what prices are predicted but also why predictions change and which factors drive price movements. Only Yin et al. (2020) partially addressed this through their attention mechanism providing feature importance insights \citep{yin2020}.

Uncertainty quantification remained largely absent from agricultural price predictions. Providing point estimates without confidence intervals or prediction intervals limits practical decision-making value, as farmers and traders need to assess risk associated with predictions for effective planning.

\section{Summary and Position of Current Research}

The literature review reveals substantial progress in agricultural price forecasting, transitioning from traditional statistical methods through machine learning approaches to sophisticated deep learning architectures. ARIMA and VAR models provided interpretable baseline approaches but struggled with non-linear patterns and large-scale data. Machine learning methods, particularly tree-based ensembles and support vector machines, demonstrated superior performance in capturing complex relationships between multiple variables. Deep learning approaches, especially LSTM and hybrid architectures, achieved state-of-the-art results by effectively modeling temporal dependencies and handling high-dimensional input spaces.

However, significant research gaps persist. Most studies focused on single vegetables or specific markets without comprehensive cross-commodity or cross-regional validation. Feature selection remained largely ad-hoc rather than systematic, and model interpretability received insufficient attention despite its importance for stakeholder adoption. The integration of diverse data sources including real-time information and broader macroeconomic factors remained limited. Finally, the gap between research models and deployable systems hindered practical impact on agricultural decision-making.

The current research addresses these gaps by: (1) developing comprehensive multivariate forecasting models for carrot prices in Dambulla market incorporating weather, fuel, supply, and market factors; (2) implementing rigorous feature selection combining Random Forest importance, Mutual Information, and Recursive Feature Elimination; (3) conducting systematic model comparison across ARIMA, univariate LSTM, multivariate LSTM, bidirectional LSTM, and Random Forest approaches; (4) performing ablation studies to quantify individual feature category contributions; (5) employing statistical validation including bootstrap confidence intervals and effect size analysis; (6) enhancing model interpretability through SHAP analysis; and (7) developing a Retrieval-Augmented Generation based AI agent to make predictions accessible and interpretable for agricultural stakeholders.

This research advances the field by providing methodologically rigorous, interpretable, and practically deployable forecasting solutions specifically tailored for Sri Lankan agricultural markets, while establishing a replicable framework applicable to other vegetables and markets in developing agricultural economies.
