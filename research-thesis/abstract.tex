\begin{abstract}

Agricultural price volatility poses significant challenges to food security, farmer livelihoods, and economic stability in developing economies. This research addresses carrot price forecasting for the Dambulla wholesale market in Sri Lanka by developing and evaluating a comprehensive forecasting system combining advanced deep learning techniques with explainable artificial intelligence. Utilizing a rich dataset of 2,013 daily observations spanning January 2020 to July 2025, the study integrates 289 initial features across six categories: historical prices, precipitation patterns from 11 growing regions, supply factors, demand indicators, fuel costs, and temporal variables. A systematic 4-stage feature selection pipeline employing Random Forest importance, Mutual Information, correlation analysis, and multicollinearity removal reduces dimensionality to 19-35 features while preserving predictive power.

Seven forecasting approaches are rigorously compared using consistent train-validation-test splits (70\%-15\%-15\%) and comprehensive evaluation metrics including Mean Absolute Percentage Error (MAPE), Mean Absolute Error (MAE), Root Mean Squared Error (RMSE), and coefficient of determination (R²). Traditional statistical methods demonstrate poor performance, with univariate ARIMA achieving MAPE exceeding 50\% and multivariate ARIMAX reaching 88.80\% MAPE, indicating fundamental inadequacy of linear assumptions for modeling complex agricultural market dynamics. Machine learning approaches show substantial improvement, with Random Forest Tuned achieving 20.84\% MAPE and R² of 0.5267.

The enhanced Bidirectional Long Short-Term Memory (LSTM) architecture emerges as the best-performing model, achieving 21.22\% test MAPE, 68.67 Rs MAE, 99.46 Rs RMSE, and R² of 0.8111. This represents a 67.4\% MAPE reduction compared to ARIMAX and demonstrates superior prediction reliability across the full price spectrum. The bidirectional architecture's ability to process sequences in both temporal directions enables effective capture of price turning points where lagged weather effects interact with recent market trends. Systematic ablation studies quantify feature category contributions: price features provide 48.7\% of total importance, weather patterns 19.2\%, and market demand 14.5\%. SHAP (SHapley Additive exPlanations) analysis enhances interpretability by revealing instance-level feature contributions and validating negative rainfall-price correlations consistent with agricultural economics theory.

The research makes several significant contributions. Methodologically, it establishes a replicable feature selection framework and fair model comparison protocol eliminating feature set bias. Technically, it demonstrates optimized bidirectional LSTM architecture outperforms both univariate and high-dimensional multivariate approaches through moderate feature selection and enhanced regularization. Empirically, it quantifies specific weather-price relationships valuable for agricultural policy formulation and establishes benchmark performance metrics for Sri Lankan vegetable forecasting. Practically, it delivers a deployment-ready system integrating the best-performing model with a Retrieval-Augmented Generation (RAG) AI agent powered by Groq API (Llama 3.3 70B), enabling natural language interaction through a Gradio web interface.

The operational system provides actionable insights for multiple stakeholder groups: farmers can optimize harvest timing based on 7-14 day price forecasts, traders can manage inventory to reduce waste, policymakers can time market interventions effectively, and consumers can plan purchases around predicted price movements. Statistical validation through bootstrap confidence intervals and 5-fold time series cross-validation confirms model stability and generalization capability.

Limitations include modest temporal coverage requiring additional historical data for capturing long-term cycles, single-market focus necessitating validation for generalization to other regions, and reduced accuracy during unprecedented extreme events falling outside training distribution. Future work encompasses multi-crop and multi-market expansion leveraging transfer learning, advanced architectures including Transformer models with attention mechanisms, probabilistic forecasting with prediction intervals, enhanced interpretability through causal inference, mobile application development for rural connectivity, and integration with agricultural extension services for wider adoption.

This research demonstrates that carefully designed machine learning systems can substantially improve agricultural price forecasting accuracy compared to traditional approaches, while maintaining interpretability through hybrid explainability strategies. The deployment-ready system with natural language interface represents a paradigm shift in agricultural intelligence accessibility, democratizing sophisticated forecasting capabilities for non-technical stakeholders. By bridging the gap between academic research and operational deployment, this work contributes to data-informed agricultural systems capable of enhancing market efficiency, protecting farmer incomes, and supporting food security in volatile agricultural markets.

\end{abstract}
