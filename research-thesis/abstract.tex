\begin{abstract}

This research develops a machine learning-based forecasting system for predicting wholesale carrot prices at the Dambulla market, Sri Lanka's largest vegetable trading center, to support informed decision-making by farmers, traders, and policymakers in managing agricultural market volatility. The study integrates 2,013 daily observations spanning January 2020 to July 2025 from multiple authoritative sources: wholesale prices from the Central Bank of Sri Lanka, precipitation data from 11 major growing regions via Copernicus Climate Data Store, fuel prices from Ceylon Petroleum Corporation, and supply-demand indicators from agricultural market reports.

Comprehensive feature engineering generated 289 variables across six categories including historical price patterns, weather conditions, regional supply factors, demand indicators, fuel costs, and temporal features. For multivariate models, a systematic 4-stage feature selection pipeline combining Random Forest importance (60\%), Mutual Information (30\%), and correlation analysis (10\%) with multicollinearity removal and dual validation reduced dimensionality to 24-35 features while preserving predictive power and ensuring fair model comparison across approaches.

Seven forecasting models were rigorously evaluated using consistent 70\%-15\%-15\% train-validation-test splits and comprehensive metrics (MAPE, MAE, RMSE, R²). Traditional statistical methods demonstrated fundamental limitations: univariate ARIMA exceeded 50\% MAPE while multivariate ARIMAX achieved 88.80\% MAPE, indicating inadequacy of linear assumptions for complex agricultural market dynamics. Machine learning approaches showed substantial improvement: Univariate LSTM achieved 21.90\% test MAPE with R² of 0.6428, Bidirectional LSTM reached 21.22\% MAPE with R² of 0.8111, Multivariate LSTM obtained 25.88\% MAPE with R² of 0.5400, Random Forest Baseline achieved 21.19\% MAPE with R² of 0.5132, and Random Forest Tuned reached 20.84\% MAPE with R² of 0.5267. Notably, the Multivariate LSTM underperformed the univariate approach, demonstrating that model architecture, regularization strength, and training strategies matter more than feature quantity alone.

Systematic ablation studies quantified feature contributions for multivariate models: price features 48.7\%, weather patterns 19.2\%, market demand 14.5\%, supply factors 10.5\%, fuel prices 10.5\%, and temporal features 5.3\%. SHAP analysis enhanced interpretability, confirming negative rainfall-price correlations consistent with agricultural economics theory. Central Highland precipitation explains 12\% of price variance while fuel costs account for 6\%, providing quantified weather-price relationships valuable for agricultural policy formulation.

The research delivers a deployment-ready system integrating LSTM forecasting capabilities with a Retrieval-Augmented Generation (RAG) AI agent powered by Groq API (Llama 3.3 70B Versatile). The natural language interface via Gradio enables farmers to optimize harvest timing, traders to manage inventory, policymakers to time interventions, and consumers to plan purchases—democratizing sophisticated forecasting for non-technical stakeholders. Statistical validation through bootstrap confidence intervals and 5-fold time series cross-validation confirmed model stability and generalization capability.

Methodological contributions include a replicable 4-stage feature selection framework and fair comparison protocol eliminating feature set bias across models. Technical contributions demonstrate optimized LSTM architectures with appropriate regularization strategies for agricultural price forecasting. Empirical contributions establish benchmark performance metrics for Sri Lankan vegetable forecasting and quantify weather-fuel-price relationships. Practical contributions provide an operational forecasting system with comprehensive documentation enabling replication for other crops and markets across Sri Lanka.

This research demonstrates that carefully designed machine learning systems substantially improve agricultural price forecasting compared to traditional statistical approaches, while maintaining interpretability through hybrid explainability strategies. By bridging academic research and operational deployment, this work contributes to data-informed agricultural intelligence capable of enhancing market efficiency, protecting farmer incomes, and supporting food security in volatile markets.

\textbf{Keywords:} carrot price prediction, LSTM neural networks, Dambulla market, agricultural forecasting, machine learning, weather-price relationships, RAG system, AI agent, time series analysis, Sri Lankan agriculture

\end{abstract}
